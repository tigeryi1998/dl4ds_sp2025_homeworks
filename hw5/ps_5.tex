\documentclass[10pt]{article}
\usepackage{amsmath, amssymb}
\usepackage{hyperref}
\begin{document}

\begin{center}
    \LARGE {Problem Set 5 - Regularization} \\[1em]
    \Large{DS542 – DL4DS} \\[0.5em]
    \large Spring, 2025 \\
    \large Sicheng Yi (Tiger Yi) \\
    \large tigeryi@bu.edu
\end{center}

\vspace{2em}

\noindent\textbf{Note:} Refer to the equations in the \textit{Understanding Deep Learning} 
textbook to solve the following problems.

\section*{AI/Correction Statement (1 point)}

You may use ChatGPT/Generative AI as a resource to help you complete the assignment. 
However, it must be used constructively to help you understand things you are 
unsure of, and be built upon with original work.

You must cite your interaction by describing your prompt and the corresponding response. 
In addition, you must  explain all output from the AI that you implement in your 
assignment. Failure to do so could result in credit deduction. 

The official GAIA Policy can be found 
\href{https://www.bu.edu/cds-faculty/culture-community/gaia-policy/}{here}.

Moreover, if this is a correction submission after the initial submission, 
you must provide a reflection on what you learned from the initial submission 
and how you corrected it.

\

\noindent Answer:

\

\noindent  For problem 9.1, I did NOT use any generate AI like the ChatGPT, I typed all the latex equations in the .tex file

\ 

\noindent  For problem 9.5, I did NOT use any generate AI like the ChatGPT, I typed all the latex equations in the .tex file

\newpage


% \vspace{3em}

\section*{Problem 9.1 (4 points)}
Consider a model where the prior distribution over the parameters is a normal distribution with mean zero and variance
$\sigma^2_{\phi}$ so that

\begin{equation}
    Pr(\phi) = \prod_{j=1}^{J} \mathcal{N}(\phi_j | 0, \sigma^2_{\phi}),
\end{equation}

\noindent where $j$ indexes the model parameters. When we apply a prior, we maximize

\begin{equation}
    \prod_{i=1}^{I} Pr(y_i | x_i, \phi) Pr(\phi).
\end{equation}

Show that the associated loss function of this model is equivalent to L2 regularization.

\ 

\noindent Answer:

\begin{equation}
     \phi^{'} =  \arg \max_{\phi} [ \prod_{i=1}^{I} Pr(y_i | x_i, \phi) Pr(\phi) ] 
\end{equation}


\begin{equation}
     \phi^{'} = - \log  \arg \min_{\phi} [ \prod_{i=1}^{I} Pr(y_i | x_i, \phi) Pr(\phi) ] 
\end{equation}

\begin{equation}
     \phi^{'} = -  \arg \min_{\phi}  \sum_{i=1}^{I} [ \log  Pr(y_i | x_i, \phi) + \log Pr(\phi) ] 
\end{equation}

\begin{equation}
    Pr(\phi) = \prod_{j=1}^{J} \mathcal{N}(\phi_j | 0, \sigma^2_{\phi}) \approx \prod_{j=1}^{J} C \cdot \exp( - \frac{\phi_j^2 }{2 \sigma^2_{\phi}} )
\end{equation}

\begin{equation}
      -  \log [ Pr(\phi) ]  =  \frac{1}{2 \sigma^2_{\phi}} \sum_{j=1}^{J} \phi_j^2 = \lambda  \sum_{j=1}^{J} \phi_j^2
\end{equation}


\begin{equation}
	- \log  Pr(y_i | x_i, \phi) =  \ell_i [\phi, x_i, y_i]  \,\ \text{negative log likelihood}
\end{equation}

\begin{equation}
     \phi^{'} = \arg \min_{\phi}  \sum_{i=1}^{I} [ \ell_i [\phi, x_i, y_i] + \lambda  \sum_{j=1}^{J} \phi_j^2 ] 
\end{equation}

\noindent Therefore, I have shown the loss function is equivalent to the L2 ridge regularization with the last term: 

$$ \lambda  \sum_{j=1}^{J} \phi_j^2 $$ 



\newpage

% \vspace{3em}

\section*{Problem 9.5 (4 points)}

Show that the weight decay parameter update with decay rate $\lambda$:

\begin{equation}
    \phi \leftarrow (1 - \lambda) \phi - \alpha \frac{\partial L}{\partial \phi},
\end{equation}

\noindent on the original loss function $L[\phi]$ is equivalent to a standard
gradient update using L2 regularization, so that the modified loss function
$\tilde{L}[\phi]$ is:

\begin{equation}
    \tilde{L}[\phi] = L[\phi] + \frac{\lambda}{2\alpha} \sum_{k} \phi_k^2,
\end{equation}

\noindent  where $\phi$ represents the parameters, and $\alpha$ is the learning rate.

\ 

\noindent Answer:

\

\noindent Take derivative of $\tilde{L}$ with respect to $\phi$


\begin{equation}
	\frac{\partial \tilde{L} }{\partial \phi} = \frac{\partial L}{\partial \phi} + \frac{\lambda}{\alpha} \phi
\end{equation}


\begin{equation}
    \phi \leftarrow  \phi - \alpha \frac{\partial \tilde{L}}{\partial \phi}
\end{equation}


\begin{equation}
    \phi \leftarrow  \phi - \alpha ( \frac{\partial L}{\partial \phi} + \frac{\lambda}{\alpha} \phi )
\end{equation}


\begin{equation}
    \phi \leftarrow  \phi - \alpha  \frac{\partial L}{\partial \phi} - \lambda \phi 
\end{equation}


\begin{equation}
    \phi \leftarrow  (1 - \lambda) \phi - \alpha  \frac{\partial L}{\partial \phi} 
\end{equation}

\

\noindent Therefore, gradient descent on the modified $\tilde{L}$ with L2 regularization

\noindent is same as applying weight decay $\lambda$ on the old loss function $L$





\end{document}
